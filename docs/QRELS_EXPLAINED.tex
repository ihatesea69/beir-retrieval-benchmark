\section{QRels Files Explained: Train, Dev, Test}

\subsection{ Quick Answer}

\textbf{3 files qrels} trong BeIR dataset là \textbf{train.tsv}, \textbf{dev.tsv}, và \textbf{test.tsv} - đây là chuẩn trong Machine Learning để:

\begin{enumerate}
  \item \textbf{train.tsv} - Huấn luyện model (nếu có supervised learning)
  \item \textbf{dev.tsv} - Validation/tuning hyperparameters
  \item \textbf{test.tsv} - Đánh giá cuối cùng (báo cáo kết quả)
\end{enumerate}

\subsection{ Statistics Overview}

\begin{lstlisting}
NFCorpus Qrels Statistics:
════════════════════════════════════════════════════════════

File         │ Lines    │ Size      │ Purpose
─────────────┼──────────┼───────────┼─────────────────────────
train.tsv    │ 110,576  │ 2,446 KB  │ Training queries (nếu cần)
dev.tsv      │ 11,386   │ 252 KB    │ Validation/tuning
test.tsv     │ 12,335   │ 273 KB    │ Final evaluation ⭐

Total: 134,297 relevance judgments
\end{lstlisting}

\subsection{1⃣ What is QRels?}

\subsubsection{Definition}

\textbf{QRels} = \textbf{Q}uery \textbf{Rel}evance Judgment\textbf{s}

File chứa \textbf{ground truth} - đánh giá thủ công của con người về độ liên quan giữa queries và documents.

\subsubsection{Format (TSV - Tab-Separated Values)}

\begin{lstlisting}[language=tsv]
query-id	corpus-id	score
PLAIN-2	MED-2427	2
PLAIN-2	MED-10	2
PLAIN-2	MED-2429	2
PLAIN-2	MED-2428	1
\end{lstlisting}

\textbf{Columns:}

\begin{itemize}
  \item \texttt{query-id}: ID của query (e.g., PLAIN-2 = "Do Cholesterol Statin Drugs Cause Breast Cancer?")
  \item \texttt{corpus-id}: ID của document (e.g., MED-10 = "Statin Use and Breast Cancer Survival")
  \item \texttt{score}: Độ liên quan
  \item \texttt{2} = Highly relevant (rất liên quan)
  \item \texttt{1} = Relevant (liên quan)
  \item \texttt{0} = Not relevant (không liên quan, thường không có trong file)
\end{itemize}

\subsection{2⃣ Why 3 Files? Train/Dev/Test Split}

\subsubsection{Machine Learning Standard Practice}

\begin{lstlisting}
┌─────────────────────────────────────────────────────────┐
│                    Full Dataset                         │
│                  134,297 judgments                      │
└─────────────────────────────────────────────────────────┘
                    ↓ Split
    ┌───────────────┼───────────────┬─────────────┐
    ↓               ↓               ↓             ↓
┌─────────┐   ┌─────────┐   ┌─────────┐   ┌─────────┐
│ Train   │   │  Dev    │   │  Test   │   │ Corpus  │
│ 82%     │   │  8%     │   │  10%    │   │         │
│110,576  │   │ 11,386  │   │ 12,335  │   │ 3,633   │
│judgments│   │judgments│   │judgments│   │documents│
└─────────┘   └─────────┘   └─────────┘   └─────────┘
     │             │             │             │
     ↓             ↓             ↓             ↓
  Training    Validation    Evaluation    Knowledge
   (learn)      (tune)       (report)       Base
\end{lstlisting}

\subsubsection{Purpose of Each Split}

\paragraph{ \textbf{train.tsv} (110,576 lines = 82\%)}

\textbf{Nhiệm vụ:} Huấn luyện supervised learning models

\textbf{Use cases:}

\begin{itemize}
  \item Train neural ranking models (BERT, T5)
  \item Learn-to-rank algorithms
  \item Fine-tune language models for domain adaptation
\end{itemize}

\textbf{Example usage:}

\begin{lstlisting}[language=python]
# Train a neural re-ranker
train_qrels = load_qrels("train.tsv")

for query, relevant_docs, irrelevant_docs in train_qrels:
    # Positive pairs
    loss += contrastive_loss(query, relevant_docs, label=1)
    
    # Negative pairs (hard negatives mining)
    loss += contrastive_loss(query, irrelevant_docs, label=0)
    
model.backward(loss)
\end{lstlisting}

\textbf{Characteristics:}

\begin{itemize}
  \item Largest split (82\% of data)
  \item Used for model optimization
  \item Can iterate multiple times (epochs)
  \item Our project: \textbf{Không dùng} (BM25/Dense are unsupervised)
\end{itemize}

\paragraph{🟢 \textbf{dev.tsv} (11,386 lines = 8\%)}

\textbf{Nhiệm vụ:} Validation và hyperparameter tuning

\textbf{Use cases:}

\begin{itemize}
  \item Tune BM25 parameters (k1, b)
  \item Select embedding model
  \item Choose chunking strategy (512 vs 1024 tokens)
  \item Tune RRF parameters (α, k)
  \item Early stopping for training
\end{itemize}

\textbf{Example usage:}

\begin{lstlisting}[language=python]
# Tune BM25 parameters
best_ndcg = 0
for k1 in [1.2, 1.5, 2.0]:
    for b in [0.5, 0.75, 1.0]:
        bm25 = BM25(k1=k1, b=b)
        results = bm25.search(dev_queries)
        ndcg = evaluate(results, dev_qrels)  # Use dev.tsv
        
        if ndcg > best_ndcg:
            best_k1, best_b = k1, b
            best_ndcg = ndcg

print(f"Best params: k1={best_k1}, b={best_b}")
\end{lstlisting}

\textbf{Characteristics:}

\begin{itemize}
  \item Medium size (8\% of data)
  \item Used during development
  \item Can be used multiple times for tuning
  \item Our project: \textbf{Có thể dùng} để tune α trong RRF
\end{itemize}

\paragraph{ \textbf{test.tsv} (12,335 lines = 10\%)}

\textbf{Nhiệm vụ:} Final evaluation và báo cáo kết quả

\textbf{Use cases:}

\begin{itemize}
  \item Đánh giá cuối cùng sau khi model đã hoàn thiện
  \item So sánh các methods (BM25 vs Dense vs Hybrid)
  \item Báo cáo metrics cho paper/thesis
  \item Benchmark trên BeIR leaderboard
\end{itemize}

\textbf{Example usage:}

\begin{lstlisting}[language=python]
# Final evaluation (ONLY ONCE!)
test_qrels = load_qrels("test.tsv")

bm25_metrics = evaluate(bm25_results, test_qrels)
dense_metrics = evaluate(dense_results, test_qrels)
hybrid_metrics = evaluate(hybrid_results, test_qrels)

print("Final Results on Test Set:")
print(f"BM25:   NDCG@10={bm25_metrics['ndcg@10']}")
print(f"Dense:  NDCG@10={dense_metrics['ndcg@10']}")
print(f"Hybrid: NDCG@10={hybrid_metrics['ndcg@10']}")
\end{lstlisting}

\textbf{ CRITICAL RULES:}

\begin{itemize}
  \item \textbf{Chỉ dùng 1 lần duy nhất} khi model đã finalized
  \item \textbf{Không được tune} dựa trên test results
  \item \textbf{Không được nhìn} test data khi develop
  \item Nếu tune based on test → \textbf{data leakage} → kết quả không valid
\end{itemize}

\textbf{Characteristics:}

\begin{itemize}
  \item Held-out set (10\% of data)
  \item Used ONLY for final reporting
  \item Our project: \textbf{Đây là file chính} để đánh giá
\end{itemize}

\subsection{3⃣ Format Details: Why TSV?}

\subsubsection{TSV (Tab-Separated Values)}

\begin{lstlisting}
Advantages:
✅ Simple text format
✅ Easy to parse (split by \t)
✅ Human-readable
✅ Works with pandas, Excel
✅ Smaller than JSON

Example:
query-id	corpus-id	score
PLAIN-2	MED-10	2
     ↑       ↑      ↑
   Tab     Tab   Newline
\end{lstlisting}

\subsubsection{Why NOT JSON/CSV?}

\textbf{JSON:}

\begin{lstlisting}[language=json]
{
  "PLAIN-2": {
    "MED-10": 2,
    "MED-2427": 2,
    ...
  }
}
\end{lstlisting}

 Larger file size  
 Harder to stream process  
 But easier for nested data

\textbf{CSV (Comma-Separated):}

\begin{lstlisting}[language=csv]
query-id,corpus-id,score
PLAIN-2,MED-10,2
\end{lstlisting}

 Commas might appear in data  
 Need escaping for text  
 But more universal

\textbf{TSV Winner:}
 Tabs rarely appear in IDs  
 No escaping needed  
 Standard for TREC/BeIR benchmarks

\subsection{4⃣ Content Analysis}

\subsubsection{Train.tsv (110,576 lines)}

\begin{lstlisting}[language=python]
# Analysis
total_lines = 110,576
queries_in_train = ~2,590  # Unique query IDs

# Average judgments per query
avg_judgments = 110,576 / 2,590 ≈ 42.7

# Score distribution
score_2 (highly relevant): ~35%
score_1 (relevant): ~65%

# Example queries in train:
- PLAIN-3, PLAIN-5, PLAIN-7, ...
- VIDEO-4, VIDEO-8, ...
\end{lstlisting}

\textbf{Observations:}

\begin{itemize}
  \item Each query has \textasciitilde{}43 relevant documents
  \item More score=1 than score=2 (easier to find relevant than highly relevant)
  \item Mix of PLAIN and VIDEO queries
\end{itemize}

\subsubsection{Dev.tsv (11,386 lines)}

\begin{lstlisting}[language=python]
# Analysis
total_lines = 11,386
queries_in_dev = ~323  # Unique query IDs

# Average judgments per query
avg_judgments = 11,386 / 323 ≈ 35.3

# Score distribution
score_2 (highly relevant): ~40%
score_1 (relevant): ~60%

# Example from dev.tsv:
PLAIN-1	MED-2421	2  # Highly relevant
PLAIN-1	MED-2422	2
PLAIN-1	MED-2414	1  # Relevant
PLAIN-1	MED-4070	1
\end{lstlisting}

\textbf{Observations:}

\begin{itemize}
  \item Smaller but still representative
  \item Used in our DATASET\_DOCUMENTATION.md examples
  \item Good for quick validation
\end{itemize}

\subsubsection{Test.tsv (12,335 lines)}

\begin{lstlisting}[language=python]
# Analysis
total_lines = 12,335
queries_in_test = ~323  # Unique query IDs

# Average judgments per query
avg_judgments = 12,335 / 323 ≈ 38.2

# Score distribution
score_2 (highly relevant): ~38%
score_1 (relevant): ~62%

# Example from test.tsv:
PLAIN-2	MED-2427	2  # "Do Cholesterol Statin Drugs Cause Breast Cancer?"
PLAIN-2	MED-10	2   # → "Statin Use and Breast Cancer Survival"
PLAIN-2	MED-2428	1
\end{lstlisting}

\textbf{Observations:}

\begin{itemize}
  \item Similar size to dev (balanced split)
  \item \textbf{This is our evaluation set} 
  \item Matches statistics in DATASET\_DOCUMENTATION.md
\end{itemize}

\subsection{5⃣ How We Use These Files}

\subsubsection{Our Project Workflow}

\begin{lstlisting}[language=python]
# data_loader.py
def load_dataset(dataset_name='nfcorpus'):
    """
    Load BeIR dataset with qrels
    """
    data_path = './data/beir_datasets/nfcorpus'
    
    # Load corpus (same for all splits)
    corpus = load_corpus(f"{data_path}/corpus.jsonl")  # 3,633 docs
    
    # Load queries (same for all splits)
    queries = load_queries(f"{data_path}/queries.jsonl")  # 3,237 queries
    
    # Load qrels - DEFAULT: test split ⭐
    qrels = load_qrels(f"{data_path}/qrels/test.tsv")  # 12,335 judgments
    
    return corpus, queries, qrels
\end{lstlisting}

\textbf{Why we use test.tsv by default:}

\begin{lstlisting}[language=python]
# Because we are NOT training:
# - BM25: Unsupervised (no training needed)
# - Dense: Pre-trained model (all-MiniLM-L6-v2)
# - No fine-tuning on NFCorpus

# Therefore:
# - train.tsv: NOT USED
# - dev.tsv: NOT USED (could use for RRF tuning)
# - test.tsv: MAIN EVALUATION SET ⭐
\end{lstlisting}

\subsubsection{Evaluation Pipeline}

\begin{lstlisting}[language=python]
# metrics.py
class RetrievalEvaluator:
    def __init__(self, qrels):
        """
        Args:
            qrels: From test.tsv
                {
                    'PLAIN-2': {
                        'MED-2427': 2,
                        'MED-10': 2,
                        'MED-2428': 1,
                        ...
                    },
                    ...
                }
        """
        self.qrels = qrels
    
    def evaluate_batch(self, results_dict):
        """
        Args:
            results_dict: Retrieved results from BM25/Dense/Hybrid
        
        Returns:
            Metrics calculated against test.tsv qrels
        """
        for query_id, results in results_dict.items():
            retrieved = [r['id'] for r in results]
            relevant = list(self.qrels[query_id].keys())  # From test.tsv
            
            ndcg10 = calculate_ndcg_at_k(retrieved, relevant, k=10)
            recall100 = calculate_recall_at_k(retrieved, relevant, k=100)
            ...
\end{lstlisting}

\subsection{6⃣ Real Examples from Files}

\subsubsection{Example 1: PLAIN-2 in test.tsv}

\begin{lstlisting}[language=tsv]
Query: PLAIN-2 = "Do Cholesterol Statin Drugs Cause Breast Cancer?"

Highly Relevant (score=2):
─────────────────────────────────────────────────────
PLAIN-2	MED-2427	2  ← "Statin use and breast cancer survival"
PLAIN-2	MED-10	2  ← "Statin Use and Breast Cancer Survival: A Nationwide Cohort Study"
PLAIN-2	MED-2429	2  ← "Statin therapy and cancer risk"
PLAIN-2	MED-2430	2  ← "Statins and cancer"
PLAIN-2	MED-2431	2  ← "Do statins have antitumor effects?"
PLAIN-2	MED-14	2  ← "Statin use after diagnosis of breast cancer"
PLAIN-2	MED-2432	2  ← "Statin use and breast cancer outcome"

Relevant (score=1):
─────────────────────────────────────────────────────
PLAIN-2	MED-2428	1  ← "Cholesterol and cancer risk"
PLAIN-2	MED-2440	1  ← "Lipid-lowering drugs and cancer"
PLAIN-2	MED-2434	1  ← "Statin therapy: potential side effects"
... (31 more documents)

Total: 7 highly relevant + 31 relevant = 38 relevant documents
\end{lstlisting}

\textbf{Why different scores?}

\begin{lstlisting}
score=2 (Highly Relevant):
✅ Directly answers the question
✅ Mentions both "statin" AND "breast cancer"
✅ Studies showing causal relationship

score=1 (Relevant):
✅ Related to topic but not direct answer
✅ Mentions only "cholesterol" or "cancer" (not both)
✅ Background information
\end{lstlisting}

\subsubsection{Example 2: Evaluation with test.tsv}

\begin{lstlisting}[language=python]
# Scenario: System retrieves for PLAIN-2
retrieved_results = [
    'MED-10',    # rank 1
    'MED-2427',  # rank 2
    'MED-2428',  # rank 3
    'MED-100',   # rank 4 (not relevant)
    'MED-2429',  # rank 5
    ...
]

# Ground truth from test.tsv
relevant_docs = {
    'MED-10': 2,     # Highly relevant
    'MED-2427': 2,
    'MED-2429': 2,
    'MED-2428': 1,   # Relevant
    ...
}

# Calculate NDCG@10
# Rank 1: MED-10 (score=2) → gain = 2 / log2(2) = 2.0
# Rank 2: MED-2427 (score=2) → gain = 2 / log2(3) = 1.262
# Rank 3: MED-2428 (score=1) → gain = 1 / log2(4) = 0.5
# Rank 4: MED-100 (not relevant) → gain = 0
# Rank 5: MED-2429 (score=2) → gain = 2 / log2(6) = 0.774

# DCG@10 = 2.0 + 1.262 + 0.5 + 0 + 0.774 = 4.536
# IDCG@10 = 2.0 + 1.262 + 1.0 + 0.861 + ... = 6.123
# NDCG@10 = 4.536 / 6.123 = 0.741 ⭐
\end{lstlisting}

\subsection{7⃣ BeIR Standard: Why This Format?}

\subsubsection{TREC/BeIR Convention}

\textbf{TREC (Text REtrieval Conference)} established this format in 1992:

\begin{lstlisting}
┌─────────────────────────────────────────────────┐
│ TREC Format (since 1992)                        │
│ query-id  Q0  doc-id  rank  score  run-id      │
│                                                  │
│ BeIR Simplified (2021):                         │
│ query-id  doc-id  relevance-score               │
└─────────────────────────────────────────────────┘
\end{lstlisting}

\textbf{Why TSV became standard:}

\begin{enumerate}
  \item Unix tools friendly: \texttt{cut}, \texttt{grep}, \texttt{awk}
  \item Large files: 100M+ judgments in TREC collections
  \item Streaming: Can process line-by-line
  \item Version control: Git diff works well
\end{enumerate}

\subsubsection{BeIR Dataset Collection (2021)}

NFCorpus is part of \textbf{BeIR benchmark} with 18 datasets:

\begin{lstlisting}
BeIR Datasets:
─────────────────────────────────────────────────
1. MS MARCO       → General web search
2. NFCorpus       → Medical/nutritional ⭐ (ours)
3. TREC-COVID     → COVID-19 research
4. FiQA           → Financial Q&A
5. SciFact        → Scientific fact verification
... (18 total)

All use same format:
- corpus.jsonl
- queries.jsonl
- qrels/{train,dev,test}.tsv ← Standard!
\end{lstlisting}

\subsection{8⃣ Common Pitfalls \& Best Practices}

\subsubsection{ \textbf{WRONG: Using test.tsv during development}}

\begin{lstlisting}[language=python]
# BAD CODE
def tune_alpha():
    test_qrels = load_qrels("test.tsv")  # ❌ NO!
    
    best_alpha = 0
    for alpha in [0.3, 0.5, 0.7]:
        results = hybrid.search(queries, alpha=alpha)
        ndcg = evaluate(results, test_qrels)  # ❌ Peeking at test!
        
        if ndcg > best:
            best_alpha = alpha  # ❌ Overfitting to test!
    
    return best_alpha
\end{lstlisting}

\textbf{Why wrong:}

\begin{itemize}
  \item Model will overfit to test set
  \item Results not generalizable
  \item Cheating the benchmark
\end{itemize}

\subsubsection{ \textbf{CORRECT: Using dev.tsv for tuning}}

\begin{lstlisting}[language=python]
# GOOD CODE
def tune_alpha():
    dev_qrels = load_qrels("dev.tsv")  # ✅ Use dev!
    
    best_alpha = 0
    for alpha in [0.3, 0.5, 0.7]:
        results = hybrid.search(dev_queries, alpha=alpha)
        ndcg = evaluate(results, dev_qrels)  # ✅ Tune on dev
        
        if ndcg > best:
            best_alpha = alpha
    
    # Final evaluation ONCE on test
    test_qrels = load_qrels("test.tsv")
    final_results = hybrid.search(test_queries, alpha=best_alpha)
    final_ndcg = evaluate(final_results, test_qrels)  # ✅ Report this
    
    return best_alpha, final_ndcg
\end{lstlisting}

\subsubsection{ \textbf{Data Leakage Prevention}}

\begin{lstlisting}[language=python]
# Split queries properly
train_queries = queries_from_train_qrels()  # 2,590 queries
dev_queries = queries_from_dev_qrels()      # 323 queries
test_queries = queries_from_test_qrels()    # 323 queries

# CRITICAL: No overlap!
assert len(set(train_queries) & set(test_queries)) == 0  # ✅
assert len(set(dev_queries) & set(test_queries)) == 0    # ✅
\end{lstlisting}

\subsection{9⃣ Our Project Usage Summary}

\subsubsection{Current Implementation}

\begin{lstlisting}[language=python]
# data_loader.py loads test.tsv by default
loader = BeirDataLoader()
corpus, queries, qrels = loader.load_dataset('nfcorpus')

# qrels comes from: ./data/beir_datasets/nfcorpus/qrels/test.tsv
# - 12,335 relevance judgments
# - 323 test queries
# - Used for FINAL evaluation

# Why test.tsv?
# - No training needed (BM25/Dense are pre-existing)
# - No hyperparameter tuning (using defaults)
# - Direct comparison of 3 methods
\end{lstlisting}

\subsubsection{Recommendations for Enhancement}

\begin{lstlisting}[language=python]
# Option 1: Use dev.tsv for RRF tuning
def find_best_rrf_params():
    """Tune α and k on dev set"""
    dev_corpus, dev_queries, dev_qrels = loader.load_dataset('nfcorpus', split='dev')
    
    best_params = None
    best_ndcg = 0
    
    for alpha in [0.3, 0.4, 0.5, 0.6, 0.7]:
        for k in [30, 60, 90]:
            hybrid = LlamaIndexHybrid(bm25, dense, alpha=alpha, k=k)
            results = hybrid.batch_search(dev_queries)
            metrics = evaluator.evaluate_batch(results)
            
            if metrics['ndcg@10'] > best_ndcg:
                best_params = (alpha, k)
                best_ndcg = metrics['ndcg@10']
    
    return best_params

# Then evaluate on test set with best params
best_alpha, best_k = find_best_rrf_params()
hybrid_final = LlamaIndexHybrid(bm25, dense, alpha=best_alpha, k=best_k)

# Load test set
test_corpus, test_queries, test_qrels = loader.load_dataset('nfcorpus', split='test')
final_results = hybrid_final.batch_search(test_queries)
final_metrics = evaluator.evaluate_batch(final_results)

print(f"Final NDCG@10 on test: {final_metrics['ndcg@10']}")
\end{lstlisting}

\subsection{ Quick Reference}

\subsubsection{File Statistics}

\begin{lstlisting}
┌──────────────┬───────────┬──────────┬─────────────────────────┐
│ File         │ Lines     │ Queries  │ Avg Judgments/Query    │
├──────────────┼───────────┼──────────┼─────────────────────────┤
│ train.tsv    │ 110,576   │ ~2,590   │ 42.7                   │
│ dev.tsv      │ 11,386    │ ~323     │ 35.3                   │
│ test.tsv     │ 12,335    │ ~323     │ 38.2 ⭐ (we use this)  │
└──────────────┴───────────┴──────────┴─────────────────────────┘
\end{lstlisting}

\subsubsection{Format}

\begin{lstlisting}[language=tsv]
query-id	corpus-id	score
PLAIN-2	MED-10	2
     ↑       ↑      ↑
  Query   Document Score
   ID       ID     (1-2)
\end{lstlisting}

\subsubsection{Purpose}

\begin{lstlisting}
train.tsv → Training (if needed)
dev.tsv   → Validation/tuning
test.tsv  → Final evaluation ⭐
\end{lstlisting}

\subsubsection{Key Rules}

\begin{lstlisting}
✅ Use dev.tsv for hyperparameter tuning
✅ Use test.tsv ONLY ONCE for final reporting
❌ Never tune based on test.tsv results
❌ Never mix train/dev/test queries
\end{lstlisting}

\subsection{ Key Takeaways}

\begin{enumerate}
  \item \textbf{3 files = standard ML practice} (train/dev/test split)
\end{enumerate}

\begin{enumerate}
  \item \textbf{TSV format} = simple, efficient, standard for IR benchmarks
\end{enumerate}

\begin{enumerate}
  \item \textbf{Score meanings}:
\end{enumerate}

\begin{itemize}
  \item 2 = Highly relevant (directly answers query)
  \item 1 = Relevant (related information)
  \item 0 = Not relevant (not in file, assumed for non-listed docs)
\end{itemize}

\begin{enumerate}
  \item \textbf{Our project uses test.tsv} because:
\end{enumerate}

\begin{itemize}
  \item No training needed (unsupervised methods)
  \item Direct evaluation and comparison
  \item Standard BeIR benchmark protocol
\end{itemize}

\begin{enumerate}
  \item \textbf{Best practice}: 
\end{enumerate}

\begin{itemize}
  \item Tune on dev.tsv (if needed)
  \item Evaluate on test.tsv (once, final)
  \item Never peek at test during development
\end{itemize}

\begin{enumerate}
  \item \textbf{File structure enables}:
\end{enumerate}

\begin{itemize}
  \item Fair comparison across methods
  \item Reproducible results
  \item Standard metrics (NDCG, Recall, MAP, MRR)
\end{itemize}