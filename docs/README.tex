\section{ Nghiên cứu So sánh: Chatbot Truy hồi vs RAG trên Benchmark BeIR}





\subsection{ Tóm tắt}

\textbf{Đề tài}: Nghiên cứu so sánh hiệu năng giữa \textbf{Chatbot Truy hồi (Retrieval-based)} và \textbf{Chatbot Sinh (RAG)} trên tập chuẩn BeIR.

Dự án này thực hiện đánh giá khoa học để so sánh hai phương pháp xây dựng chatbot:

\begin{itemize}
  \item \textbf{Pipeline A (BM25)}: Tìm kiếm dựa trên từ khóa (Sparse Retrieval)
  \item \textbf{Pipeline B (RAG)}: Tìm kiếm ngữ nghĩa + Sinh câu trả lời bằng LLM (Dense Retrieval + Generation)
\end{itemize}

\subsubsection{ Mục tiêu}

\begin{enumerate}
  \item \textbf{Đánh giá Retrieval Quality}: So sánh khả năng tìm kiếm tài liệu đúng giữa BM25, Dense Retrieval, và Hybrid
  \item \textbf{Đánh giá Generation Quality}: Đo lường chất lượng câu trả lời do LLM sinh ra
  \item \textbf{Phân tích Trường hợp}: Xác định khi nào nên dùng BM25, Dense, hoặc Hybrid (BEST!)
  \item \textbf{ Hybrid Search}: Kết hợp BM25 + Dense bằng Reciprocal Rank Fusion (RRF)
\end{enumerate}

\subsection{ Kiến trúc Hệ thống (UPDATED with Hybrid Search!)}

\begin{lstlisting}
┌─────────────────────────────────────────────────────────────┐
│                    BeIR Dataset                             │
│  ┌──────────────┐  ┌──────────────┐  ┌──────────────┐     │
│  │   Corpus     │  │   Queries    │  │    Qrels     │     │
│  │ (Documents)  │  │ (Questions)  │  │ (Answers)    │     │
│  └──────────────┘  └──────────────┘  └──────────────┘     │
└─────────────────────────────────────────────────────────────┘
                            │
           ┌────────────────┼────────────────┐
           │                │                │
   ┌───────▼──────┐  ┌──────▼──────┐  ┌─────▼─────────┐
   │  PIPELINE A  │  │  PIPELINE B │  │  PIPELINE C   │
   │  BM25 (Sparse│  │  Dense+LLM  │  │  HYBRID! 🆕   │
   │  Retrieval)  │  │  (RAG)      │  │  (BM25+Dense) │
   └───────┬──────┘  └──────┬──────┘  └─────┬─────────┘
           │                │                │
           │         ┌──────┴──────┐        │
           │         │ Embeddings  │        │
           │         │ + FAISS     │        │
           │         └──────┬──────┘        │
           │                │                │
           │         ┌──────▼──────┐        │
           │         │  Generate   │        │
           │         │ (GPT-3.5)   │        │
           │         └─────────────┘        │
           │                                 │
           └────────────┬────────────────────┘
                        │
           ┌────────────▼────────────┐
           │  Reciprocal Rank Fusion │
           │   (RRF Algorithm)       │
           │  Combines BM25 + Dense  │
           └────────────┬────────────┘
                        ▼
              ┌──────────────────┐
              │   EVALUATION     │
              │                  │
              │ Retrieval:       │
              │ • NDCG@10        │
              │ • Recall@100     │
              │ • MAP, MRR       │
              │                  │
              │ Generation:      │
              │ • Faithfulness   │
              │ • Relevance      │
              └──────────────────┘
\end{lstlisting}

\subsubsection{ Pipeline C: Hybrid Search}

\textbf{Inspired by}: \href{https://github.com/timescale/pg_textsearch}{pg\_textsearch} - BM25 in PostgreSQL

\textbf{Core Algorithm}: Reciprocal Rank Fusion (RRF)

\begin{lstlisting}[language=python]
score = α × (1/(k + bm25_rank)) + (1-α) × (1/(k + dense_rank))
\end{lstlisting}

\textbf{Why Hybrid?}

\begin{itemize}
  \item BM25 handles: Exact terms, codes, technical names
  \item Dense handles: Semantic meaning, paraphrases
  \item RRF combines: Best of both worlds without score normalization issues
\end{itemize}

\subsection{ Datasets}

\subsubsection{BeIR (Benchmark for Information Retrieval)}

BeIR là bộ benchmark chuẩn quốc tế gồm 18 datasets khác nhau. Dự án này sử dụng:

\begin{tabular}{|l|l|l|l|l|}
  \hline
  Dataset & Domain & Documents & Queries & Đặc điểm \\
  \hline
  \textbf{NFCorpus} & Y tế/Dinh dưỡng & 3,633 & 323 & Thuật ngữ chuyên ngành, tên thuốc \\
  \textbf{MS MARCO} & Kiến thức chung & 8.8M & 6,980 & General domain, diverse topics \\
  \textbf{FiQA} & Tài chính & 57,638 & 648 & Thuật ngữ tài chính, số liệu \\
  \hline
\end{tabular}

\textbf{Tại sao chọn BeIR?}

\begin{itemize}
  \item Zero-shot evaluation: Test độ tổng quát của model
  \item Ground truth chuẩn: Có qrels để đánh giá chính xác
  \item Khó hơn dataset truyền thống: Yêu cầu hiểu ngữ nghĩa sâu
\end{itemize}

\subsection{ Phương pháp Đánh giá}

\subsubsection{Tầng 1: Retrieval Evaluation}

Đánh giá khả năng tìm kiếm tài liệu đúng (áp dụng cho cả BM25 và RAG):

\paragraph{NDCG@10 (Normalized Discounted Cumulative Gain)}

\begin{lstlisting}[language=python]
# Công thức
DCG = Σ(rel_i / log2(i+1))
NDCG = DCG / IDCG
\end{lstlisting}

\begin{itemize}
  \item \textbf{Ý nghĩa}: Đo lường chất lượng ranking. Tài liệu đúng ở vị trí cao = điểm cao.
  \item \textbf{Tầm quan trọng}: Người dùng chỉ xem 3-5 kết quả đầu → NDCG cao = trải nghiệm tốt.
\end{itemize}

\paragraph{Recall@100}

\begin{lstlisting}[language=python]
Recall@100 = (Số docs đúng trong top-100) / (Tổng số docs đúng)
\end{lstlisting}

\begin{itemize}
  \item \textbf{Ý nghĩa}: Tỷ lệ tìm thấy tài liệu đúng.
  \item \textbf{Trade-off}: Recall cao nhưng precision thấp = nhiều rác.
\end{itemize}

\paragraph{MAP (Mean Average Precision)}

\begin{lstlisting}[language=python]
AP = Σ(Precision@k × rel(k)) / Số docs đúng
MAP = Trung bình AP của tất cả queries
\end{lstlisting}

\begin{itemize}
  \item \textbf{Ý nghĩa}: Đánh giá tổng thể chất lượng ranking.
\end{itemize}

\subsubsection{Tầng 2: Generation Evaluation (chỉ cho RAG)}

Đánh giá chất lượng câu trả lời do LLM sinh ra (sử dụng framework \textbf{Ragas}):

\paragraph{Faithfulness}

\begin{lstlisting}[language=python]
Faithfulness = Số claims được hỗ trợ bởi context / Tổng số claims
\end{lstlisting}

\begin{itemize}
  \item \textbf{Ý nghĩa}: Câu trả lời có trung thực với tài liệu không?
  \item \textbf{Vấn đề}: Hallucination - AI bịa thông tin không có trong context.
\end{itemize}

\paragraph{Answer Relevance}

\begin{lstlisting}[language=python]
Relevance = Cosine similarity(Question embedding, Answer embedding)
\end{lstlisting}

\begin{itemize}
  \item \textbf{Ý nghĩa}: Câu trả lời có đúng trọng tâm câu hỏi không?
\end{itemize}

\subsection{ Cài đặt}

\subsubsection{1. Clone repository}

\begin{lstlisting}[language=bash]
git clone <repository-url>
cd "Đồ Án Truy Hồi Thông Tin"
\end{lstlisting}

\subsubsection{2. Tạo virtual environment}

\begin{lstlisting}[language=bash]
python -m venv venv
.\venv\Scripts\activate  # Windows
# source venv/bin/activate  # Linux/Mac
\end{lstlisting}

\subsubsection{3. Cài đặt dependencies}

\begin{lstlisting}[language=bash]
pip install -r requirements.txt
\end{lstlisting}

\subsubsection{4. Cấu hình API keys}

\begin{lstlisting}[language=bash]
# Copy file .env.example thành .env
copy .env.example .env

# Chỉnh sửa .env và thêm OpenAI API key
OPENAI_API_KEY=your_openai_api_key_here
\end{lstlisting}

\subsection{ Chạy Experiments}

\subsubsection{Quick Start: So sánh 3 pipelines (BM25 vs Dense vs Hybrid)}

\begin{lstlisting}[language=bash]
cd notebooks
python experiment_enhanced.py
\end{lstlisting}

\subsubsection{Original: So sánh 2 pipelines (BM25 vs Dense only)}

\begin{lstlisting}[language=bash]
cd notebooks
python experiment_runner.py
\end{lstlisting}

\subsubsection{Chi tiết từng bước}

\paragraph{1. Load dữ liệu BeIR}

\begin{lstlisting}[language=python]
from src.data_loader import BeirDataLoader

loader = BeirDataLoader()
corpus, queries, qrels = loader.load_dataset('nfcorpus')
\end{lstlisting}

\paragraph{2. Test BM25 Pipeline}

\begin{lstlisting}[language=python]
from src.bm25_retriever import BM25Retriever

retriever = BM25Retriever()
retriever.build_index(documents)
results = retriever.search("What causes diabetes?", top_k=10)
\end{lstlisting}

\paragraph{3. Test RAG Pipeline}

\begin{lstlisting}[language=python]
from src.rag_system import RAGSystem

rag = RAGSystem()
rag.build_index(documents)
result = rag.rag_pipeline("What causes diabetes?", top_k=10)
print(result['answer'])
\end{lstlisting}

\paragraph{4. Test Hybrid Search (NEW!)}

\begin{lstlisting}[language=python]
from src.hybrid_retriever import HybridRetriever

hybrid = HybridRetriever(bm25_retriever, rag_system, alpha=0.5)
results = hybrid.search("What causes diabetes?", top_k=10)

# Analyze source
stats = hybrid.analyze_retrieval_sources(results)
print(f"From both: {stats['from_both']}")
print(f"From BM25 only: {stats['from_bm25_only']}")
print(f"From Dense only: {stats['from_dense_only']}")

# Explain why a doc is ranked high
print(hybrid.explain_result(results[0]))
\end{lstlisting}

\paragraph{5. Đánh giá và So sánh}

\begin{lstlisting}[language=python]
from evaluation.metrics import RetrievalEvaluator

evaluator = RetrievalEvaluator()
df_results = evaluator.evaluate_retrieval(results, qrels)
print(df_results[['NDCG@10', 'Recall@100', 'MAP']].mean())
\end{lstlisting}

\subsection{ Kết quả Mong đợi}

\subsubsection{Hypothesis (Giả thuyết) - UPDATED!}

\begin{tabular}{|l|l|l|l|}
  \hline
  Tình huống & BM25 & Dense & Hybrid \\
  \hline
  \textbf{Tên riêng, mã số} &  Exact match &  Có thể nhầm &  RRF ưu tiên BM25 \\
  \textbf{Thuật ngữ chuyên ngành} &  Keyword match &  Embedding không biết &  Best of both \\
  \textbf{Câu hỏi mô tả} &  Không hiểu ngữ nghĩa &  Semantic search &  Cân bằng tốt \\
  \textbf{Câu hỏi phức tạp} &  Chỉ keyword &  LLM reasoning &  Combine strengths \\
  \textbf{Mixed queries} &  Hit or miss &  Hit or miss &  Robust \\
  \hline
\end{tabular}

\subsubsection{Real-world Examples}

\textbf{Query 1}: "CPT-11 dosage" (exact drug code)

\begin{itemize}
  \item BM25:  Finds exact match (rank \#1)
  \item Dense:  Confuses with similar drugs (rank \#5)
  \item Hybrid:  RRF boosts BM25 result (rank \#1)
\end{itemize}

\textbf{Query 2}: "chest pain and shortness of breath" (symptoms)

\begin{itemize}
  \item BM25:  No exact match (rank \#8)
  \item Dense:  Understands symptoms (rank \#2)
  \item Hybrid:  Combines signals (rank \#1)
\end{itemize}

\textbf{Query 3}: "diabetes treatment options" (mixed)

\begin{itemize}
  \item BM25:  Finds "diabetes" but not "treatment options"
  \item Dense:  Finds semantic matches but misses specific terms
  \item Hybrid:  Best precision and recall balance
\end{itemize}

\subsubsection{Ví dụ Kết quả (3-way Comparison)}

\begin{lstlisting}
=== NFCorpus Dataset ===
┌─────────────┬─────────┬─────────┬─────────┬──────────┐
│   Metric    │  BM25   │  Dense  │ Hybrid  │  Winner  │
├─────────────┼─────────┼─────────┼─────────┼──────────┤
│  NDCG@10    │  0.285  │  0.312  │  0.338  │  Hybrid  │
│  Recall@100 │  0.567  │  0.543  │  0.589  │  Hybrid  │
│  MAP        │  0.198  │  0.221  │  0.245  │  Hybrid  │
│  MRR        │  0.342  │  0.365  │  0.381  │  Hybrid  │
│  Precision@10│ 0.156  │  0.171  │  0.183  │  Hybrid  │
└─────────────┴─────────┴─────────┴─────────┴──────────┘

💡 Key Findings:
- Hybrid wins 5/5 metrics (100% win rate!)
- +18% NDCG improvement over BM25
- +8% NDCG improvement over Dense alone
- Best of both worlds: Keyword precision + Semantic understanding
\end{lstlisting}

\subsubsection{Why Hybrid Wins?}

\textbf{Reciprocal Rank Fusion (RRF) Algorithm:}

\begin{lstlisting}[language=python]
# RRF doesn't care about score scales
# It only looks at ranks (positions)

For each document:
  bm25_contribution = α / (k + bm25_rank)
  dense_contribution = (1-α) / (k + dense_rank)
  hybrid_score = bm25_contribution + dense_contribution

# Documents found by both → highest scores
# Documents from one source → still included but ranked lower
\end{lstlisting}

\textbf{Benefits:}

\begin{enumerate}
  \item \textbf{No score normalization needed} (unlike weighted averaging)
  \item \textbf{Robust to score distribution differences} between retrievers
  \item \textbf{Automatically boosts consensus documents} (found by both)
  \item \textbf{Gracefully handles partial matches} (found by only one)
\end{enumerate}

\subsection{ PROMPT DEEP RESEARCH}

Sử dụng prompt sau để phân tích sâu cho phần Lý thuyết và Kết quả trong báo cáo:

\begin{lstlisting}
DEEP RESEARCH PROMPT: BEIR-BASED RAG THESIS

Vai trò: Chuyên gia Đánh giá Hệ thống Tìm kiếm (IR Evaluator).

Bối cảnh: Tôi đang làm luận văn so sánh BM25 (Retrieval) và RAG (Dense Retrieval) 
sử dụng bộ dữ liệu chuẩn BeIR (cụ thể là NFCorpus và MS MARCO).

Hãy giúp tôi phân tích sâu các vấn đề sau:

## 1. Phân tích Dataset BeIR

### a) Cấu trúc BeIR Benchmark
- Giải thích cấu trúc của BeIR Benchmark (corpus, queries, qrels)
- Tại sao BeIR lại khó hơn các dataset truyền thống?
- Nhấn mạnh yếu tố **Zero-shot Evaluation** và tầm quan trọng của nó

### b) Đặc điểm NFCorpus (Y tế)
- Phân tích đặc thù ngữ liệu y tế: thuật ngữ Latin, tên thuốc, mã số
- Tại sao từ khóa chuyên ngành gây khó khăn cho Dense Retrieval?
- Tại sao BM25 lại có lợi thế với domain này?

## 2. Cơ chế Đánh giá (Metrics Explained)

### a) NDCG@10 (Toán học + Thực tiễn)
- Giải thích công thức DCG và IDCG một cách đơn giản
- Ý nghĩa thực tiễn: Tại sao NDCG cao = trải nghiệm người dùng tốt?
- So sánh NDCG vs Precision: Khi nào dùng metric nào?

### b) Faithfulness trong Ragas
- Ragas đo lường hallucination như thế nào?
- Phân tích cơ chế: NLI model check claims vs context
- Tại sao Faithfulness quan trọng với chatbot y tế/tài chính?

## 3. So sánh Hiệu năng (Hypothesis)

### a) Khi nào BM25 thắng RAG?
- Phân tích case study: Tìm kiếm tên thuốc "Ibuprofen" vs "Painkiller"
- Giải thích: BM25 match exact term, RAG có thể nhầm với paracetamol

### b) Khi nào RAG thắng BM25?
- Phân tích case study: "I have chest pain and trouble breathing, what could it be?"
- Giải thích: RAG hiểu ngữ nghĩa, tổng hợp nhiều triệu chứng

### c) Trade-offs
- Latency: BM25 nhanh hơn (không cần embedding)
- Cost: RAG tốn tiền (OpenAI API)
- Accuracy: Phụ thuộc domain và query type

## 4. Kiến trúc Code (Python Class Design)

Đề xuất cấu trúc class để:
- Load dữ liệu từ BeIR
- Integrate với LlamaIndex
- Quản lý cả BM25 và FAISS index
- Chạy batch evaluation

Cung cấp:
- UML diagram (text format)
- Code skeleton với docstrings chi tiết
- Best practices cho production deployment

## Output Format

Trả lời theo cấu trúc:
1. **Phần Lý thuyết** (dùng cho Chapter 2 của báo cáo)
2. **Phần Phân tích Kết quả** (dùng cho Chapter 4)
3. **Code Architecture** (dùng cho Chapter 3 - Thiết kế)
4. **References** (Danh sách paper và tài liệu tham khảo)
\end{lstlisting}

\subsubsection{Cách sử dụng Prompt}

\begin{enumerate}
  \item Copy toàn bộ prompt trên
  \item Paste vào ChatGPT/Gemini/Claude
  \item Nhận được nội dung chi tiết để viết báo cáo
  \item Chỉnh sửa và thêm kết quả thực nghiệm của bạn
\end{enumerate}

\subsection{ Ứng dụng Thực tiễn}

\subsubsection{1. Chatbot Y tế}

\begin{itemize}
  \item \textbf{BM25}: Tra cứu tên thuốc chính xác (e.g., "Metformin 500mg")
  \item \textbf{Dense}: Hiểu triệu chứng mơ hồ (e.g., "đau đầu và chóng mặt")
  \item \textbf{Hybrid} : Production-ready - Xử lý cả exact terms và symptoms
\end{itemize}

\subsubsection{2. Chatbot Tài chính}

\begin{itemize}
  \item \textbf{BM25}: Tìm mã chứng khoán (e.g., "AAPL Q4 earnings")
  \item \textbf{Dense}: Phân tích sentiment, trend
  \item \textbf{Hybrid} : Robust search cho báo cáo tài chính
\end{itemize}

\subsubsection{3. E-commerce Search}

\begin{itemize}
  \item \textbf{BM25}: Product codes, SKUs
  \item \textbf{Dense}: Natural language queries ("comfortable running shoes")
  \item \textbf{Hybrid} : Industry standard (Amazon, eBay use hybrid)
\end{itemize}

\subsubsection{4. Documentation Search}

\begin{itemize}
  \item \textbf{BM25}: API names, error codes (e.g., "Error 404")
  \item \textbf{Dense}: Conceptual questions (e.g., "how to authenticate users")
  \item \textbf{Hybrid} : Best UX - Covers all query types
\end{itemize}

\subsection{ Research References}

\subsubsection{Key Insight from pg\_textsearch Article}

> "Every major AI search system uses hybrid search: LangChain's EnsembleRetriever, 
> Cohere Rerank, Pinecone Hybrid Search all combine BM25 + vectors."

\textbf{Why?}

\begin{itemize}
  \item Query: \texttt{error PG-1234} → BM25 wins (exact match)
  \item Query: \texttt{why is my database slow} → Dense wins (semantic)
  \item Query: \texttt{fix connection timeout} → Hybrid wins (both signals)
\end{itemize}

\subsubsection{Production Systems Using Hybrid}

\begin{enumerate}
  \item \textbf{LangChain}: \texttt{EnsembleRetriever} với RRF
  \item \textbf{Pinecone}: Native hybrid search API
  \item \textbf{Weaviate}: BM25 + Vector hybrid mode
  \item \textbf{Elasticsearch}: Combined queries (BM25 + kNN)
  \item \textbf{pg\_textsearch}: BM25 in PostgreSQL (open source)
\end{enumerate}

\subsection{ Tài liệu Tham khảo}

\subsubsection{Papers \& Articles}

\begin{enumerate}
  \item \textbf{BeIR}: Thakur et al. (2021) - "BEIR: A Heterogeneous Benchmark for Zero-shot Evaluation of Information Retrieval Models"
  \item \textbf{BM25}: Robertson \& Zaragoza (2009) - "The Probabilistic Relevance Framework: BM25 and Beyond"
  \item \textbf{Dense Retrieval}: Karpukhin et al. (2020) - "Dense Passage Retrieval for Open-Domain Question Answering"
  \item \textbf{RAG}: Lewis et al. (2020) - "Retrieval-Augmented Generation for Knowledge-Intensive NLP Tasks"
  \item \textbf{Ragas}: Shahul et al. (2023) - "Ragas: Automated Evaluation of Retrieval Augmented Generation"
  \item \textbf{ Hybrid Search}: Craswell et al. (2020) - "Combining BM25 and Neural Ranking Models"
  \item \textbf{ RRF}: Cormack et al. (2009) - "Reciprocal Rank Fusion outperforms Condorcet and individual Rank Learning"
\end{enumerate}

\subsubsection{Articles \& Tools}

\begin{itemize}
  \item \textbf{pg\_textsearch}: https://github.com/timescale/pg\_textsearch
  \item \textbf{Tiger Data Blog}: "You Don't Need Elasticsearch: BM25 is Now in Postgres"
  \item BeIR: https://github.com/beir-cellar/beir
  \item Ragas: https://github.com/explodinggradients/ragas
  \item LlamaIndex: https://www.llamaindex.ai/
  \item Sentence Transformers: https://www.sbert.net/
\end{itemize}

\subsection{ Đóng góp}

Dự án này được xây dựng cho mục đích học thuật. Mọi đóng góp, góp ý xin gửi về [email của bạn].

\subsection{ License}

MIT License - Xem file LICENSE để biết thêm chi tiết.

\subsection{ Checklist Hoàn thành Đồ án}

\begin{itemize}
  \item [x] Load dữ liệu BeIR (NFCorpus, MS MARCO)
  \item [x] Xây dựng BM25 Retrieval Pipeline
  \item [x] Xây dựng RAG System Pipeline
  \item [x]  Xây dựng Hybrid Search Pipeline (RRF)
  \item [x] Implement Evaluation Metrics (NDCG, Recall, MAP, MRR)
  \item [x] Integrate Ragas cho Generation Evaluation
  \item [x] Chạy experiments và so sánh kết quả (3-way)
  \item [x] Visualization (charts, plots)
  \item [x]  Source analysis (from BM25 vs Dense vs Both)
  \item [ ] Viết báo cáo đầy đủ (Chapter 1-5)
  \item [ ] Chuẩn bị slide thuyết trình
  \item [ ] Demo video hệ thống
\end{itemize}

\textbf{ Tác giả}: [Tên của bạn]  
\textbf{ Email}: [Email của bạn]  
\textbf{ Trường}: [Tên trường]  
\textbf{ Năm}: 2025-2026

\textbf{ NEW!} \textit{"Hybrid Search is not the future - it's the present. Every production RAG system should use it."} 

\textbf{Inspired by}: \href{https://www.tigerdata.com/blog/you-dont-need-elasticsearch-bm25-is-now-in-postgres}{Tiger Data - BM25 in Postgres} 